\chapter*{Kurzfassung}
\begingroup
\begin{table}[h!]
\setlength\tabcolsep{0pt}
\begin{tabular}{p{3.7cm}p{11.7cm}}
%Titel: & \DerTitelDerArbeit \\
Verfasser/in: & \DerAutorDerArbeit \\
Kurs: & \DieKursbezeichnung \\
Ausbildungsstätte: & \DerNameDerFirma\\
\end{tabular}
\end{table}
\endgroup

Folgende Arbeit beschreibt ein Konzept für eine graphbasierte Service-Registry. Dabei wurde eine Gegenüberstellung zwischen dem hergeleiteten Konzept der Service-Registry und einer Komposition verschiedener Produkte zur Lösung desselben Problems durchgeführt. Ein Ergebnis dieser Gegenüberstellung war, dass sich die unterschiedlichen Produkte sehr gut ergänzen können, da sie verschiedene Probleme lösen.\\ Zu Beginn der Arbeit wurde noch eine Defintion von Microservices auf Basis ihrer Eigenschaften vorgenommen. Des weiteren wurden wichtige Konzepte von Microservices, \textit{Observability} und \textit{Governance}, beschrieben, um eine klare Basis zu schaffen auf der Kriterien definiert wurden, die in dem Vergleich genutzt wurden. Außerdem wurde mithilfe von Grundlagen aus der Graphentheorie erläutert, wieso Graphen und im speziellen Flüsse eine passende Datenstruktur zur Darstellung von Abhängigkeiten zwischen Services sind. Dabei wurde sowohl eine Kapazitätsfunktion, als auch eine Flussfunktion zur mathematischen Repräsentation des Graphen definiert.


