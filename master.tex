\documentclass[
	12pt,
	BCOR=5mm,
	DIV=12,
	headinclude=on,
	footinclude=off,
	parskip=half,
	bibliography=totoc,
	listof=entryprefix,
	toc=listof,
	numbers=noenddot,
	plainfootsepline
]{scrreprt}

%	Konfigurationsdatei einziehen
\input{config}

\begin{document}

\TitelDerArbeit{TODO}
\AutorDerArbeit{Aaron Schweig}
\Firma{Hays AG}
\Kurs{WWI18SEC}

\input{titlepage}

\pagenumbering{roman} % Römische Seitennummerierung
\normalfont

%	Kurzfassung
\chapter*{Kurzfassung}
\begingroup
\begin{table}[h!]
\setlength\tabcolsep{0pt}
\begin{tabular}{p{3.7cm}p{11.7cm}}
%Titel: & \DerTitelDerArbeit \\
Verfasser/in: & \DerAutorDerArbeit \\
Kurs: & \DieKursbezeichnung \\
Ausbildungsstätte: & \DerNameDerFirma\\
\end{tabular}
\end{table}
\endgroup

Folgende Arbeit beschreibt ein Konzept für eine graphbasierte Service-Registry. Dabei wurde eine Gegenüberstellung zwischen dem hergeleiteten Konzept der Service-Registry und einer Komposition verschiedener Produkte zur Lösung desselben Problems durchgeführt. Ein Ergebnis dieser Gegenüberstellung war, dass sich die unterschiedlichen Produkte sehr gut ergänzen können, da sie verschiedene Probleme lösen.\\ Zu Beginn der Arbeit wurde noch eine Defintion von Microservices auf Basis ihrer Eigenschaften vorgenommen. Des weiteren wurden wichtige Konzepte von Microservices, \textit{Observability} und \textit{Governance}, beschrieben, um eine klare Basis zu schaffen auf der Kriterien definiert wurden, die in dem Vergleich genutzt wurden. Außerdem wurde mithilfe von Grundlagen aus der Graphentheorie erläutert, wieso Graphen und im speziellen Flüsse eine passende Datenstruktur zur Darstellung von Abhängigkeiten zwischen Services sind. Dabei wurde sowohl eine Kapazitätsfunktion, als auch eine Flussfunktion zur mathematischen Repräsentation des Graphen definiert.




%	Inhaltsverzeichnis
\tableofcontents

%	Abbildungsverzeichnis
\listoffigures

%	Tabellenverzeichnis
% \listoftables

% 	Abkürzungsverzeichnis (siehe Datei acronyms.tex!)
\clearpage
\chapter*{Abkürzungsverzeichnis}	
\addcontentsline{toc}{chapter}{Abkürzungsverzeichnis}


\begin{acronym}
	\acro{DHBW}{Duale Hochschule Baden-Württemberg}
	\acro{AWS}{Amazon Web Services}
	\acro{GCP}{Google Cloud Platform}
	\acro{CNCF}{Cloud Native Compute Foundation}
	\acro{RDBMS}{Relational Database Management System}
	\acro{BMBF}{Bundesministerium für Bildung und Forschung}	
	\acro{SOA}{Service Orientated Architechture}
	\acro{ESB}{Enterprise Service Bus}
	\acro{DBMS}{Datenbank Management System}
	\acro{APM}{Application Performance Monitoring}
	\acro{ML}{Machine Learning}
\end{acronym}

\ohead{Acronyms} % Neue Header-Definition

%--------------------------------
% Start des Textteils der Arbeit
%--------------------------------
\clearpage
\ihead{\chaptername~\thechapter}
\ohead{\headmark}
\pagenumbering{arabic}

Outline:
\begin{itemize}
	\item Einleitung schreiben, wieso das Thema wichtig ist und wieso die Schnittstelle so relevant ist
	\begin{itemize}
		\item kurzes Aufzeigen des Problems und Hintergrundes der Arbeit
	\end{itemize}
	\item Kurz die Historie von Microservices erklären
	\begin{itemize}
		\item von Monolithen über SOA bis hin zu Microservices
		\item Anhand eines versimplifizierten beispiels aufzeigen wie Zielarchitekturen aussehen
	\end{itemize}
	\item Def. Microservice, Gouvernance und Observability raushauen
	\begin{itemize}
		\item Dabei auf die verschiedenen Säulen der Observability achten
		\item Wichtigstens Aspekte bei Gouvernance noch einmal herausstellen
		\item Eigentliche Pros der Microservice-Def unter den Gesichtspunkten der G\&O evlt nicht mehr so gut
	\end{itemize}
	\item Wo liegt überhaupt das Problem (nochmal aus Einleitung aufgreifen), Herausarbeiten, evtl. ja schon passiert?
	\item Herausarbeiten von Kriterien für ein Tool im Schnittstellenbereich zwischen G\&O
	\item Wie sehen potenzielle Lösungen unter Berücksichtigung der auf Basis des Problems hergeleiteten Kriterien aus?
	\begin{itemize}
		\item Eine Idee von Elastic vorstellen $\Rightarrow$ APM zusammen mit ML-Nodes können zur Anomaly-Detection und Predicitve Maintance genutzt werden. Lösen die das eigentliche Problem, wenn ja in welchem Ausmaß
		\item Kurze Einführung in Graphdatenbanken, im speziellen Neo4J. Wieso passen die so gut? referenzieren auf Bilder und \enquote{Abhängigkeiten} Wie kann damit eine Lösung konstruiert werden?
		\begin{itemize}
			\item Statischer Ansatz $\Rightarrow$ von den Entwicklern definierte Abhängigkeiten werden in der Service-Definition angegeben. Der Graph kann aus diesen SD's erstellt werden. Hier gibt es Probleme: Manuelle Veranwtortlichkeit widerspricht eigentlicher Idee von Risikoeinschätzung, da ein neuer Fehler Mensch eingebaut wird.
			\item Automatischer Nutzungsbasierter, somit dynamischer Aufbau des Graphen. Prinzip ähnlich wie bei Prometheus (populäres Tool zum sammeln von Metriken). Konzept ähnlich wie Distributed Tracing. Dort werden alle Traces mit ihren Spans zusammengefasst und zentral ausgewertet. Kombination aus dieser Idee mit Prometheus-Ansatz sieht dann folgendermaßen aus:
			
			Jeder Service exposed einen \texttt{/traffic}-Endpoint der unstrukturierte Daten bzgl. der Netzwerkaktivität enthält. Diese können dann in einem Service der für den Aufbau des Graphen zuständig ist zu ebendiesem umgewandelt werden.
			\item Es kann zusätzlich überlegt werden, ob anstatt dieses \texttt{PULL}-Verfahrens eine Push Variante gewählt wird, welche ähnlich wie \texttt{fluentd} als Sidecar in einer Containererisierten Umgebung läuft. Die könnte dann periodisch die Netzwerkdaten pushen, wenn sie Zugriff darauf hat.
		\end{itemize}
		\item Es kann überlegt werden welche Metadaten zusätzlich im Rahmen dieses Graphen eine sinnvolle Ergänzugn bieten würden, sodass ein weiterer Mehrwert geschaffen werden kann. (Ist das noch im Rahmen der Arbeit? Kommt das eher in den Ausbilck mit ein paar Ideen?)
	\end{itemize}
	\item Wie können jetzt die Sachen in dem coolen Graphen genutzt werden?
	\begin{itemize}
		\item Kurze \texttt{CYPHER-QUERIES} Aufschreiben:
		\begin{itemize}
			\item Wie finde ich affectete services raus
			\item was passiert wenn ein Service X ausfällt
			\item Was ist die Zentralste Komponente in meinem Unternehmen
			\item usw.
		\end{itemize}
	\end{itemize}
\end{itemize}

Fragen:
\begin{itemize}
	\item Muss ich erklären was ein Graph ist und wie er funktioniert?
\end{itemize}

\chapter{Einleitung}

\chapter{Microservice - Governance - Observability}

Die Microservicearchitektur wurde erstmals 20XX von XXXX \marginpar{TODO: Quelle finden} vorgeschlagenen. Dabei wurden wichtige Elemente aus dem Vorgänger, der \ac{SOA} übernommen. Außerdem stellt diese Architektur einen Nachfolger für den monolithischen Architekturansatz dar. Im folgenden Teil soll anhand eines kurzes Beispiels erläutert werden, welche Aspekte der einzelnen Ansätze in die Microservicearchitektur einspielen.

\section{Die Geschichte der Microservicearchitektur}

Applikationen dienen als Automatisierer und sollen helfen komplexe Prozesse einfacher und am besten ohne menschliches Zutun beenden zu können. Dabei besitz die Applikation Aufgaben, die aus der Gesamtheit des Prozesses erwachsen. Es kann also sein, dass Daten aus vielen verschiedenen Bereichen eines Unternehmens benötigt werden. Anhand eines vereinfachten Prozesses sollen nun die verscheidenen Architekturen abgeleitet werden. Im Beispiel wird ein Online-Shop dargestellt.

\begin{figure}[h]
	\centering
	\includegraphics[width=1.0\linewidth]{img/prozess_eCommernce.png}
	\caption[Prozess Online-Shop]{Vereinfachter Prozess eines Online-Shops mit vier Komponenten\\ Quelle: Eigen}
	\label{fig:prozess_online_shop}
\end{figure}

In Abbildung \vref{fig:prozess_online_shop} ist der Prozess beschrieben. Es wird dabei nur ein einfacher Bestellprozess betrachtet. Ein Kunde bestellt auf einer Website bestimmte Produkte. Im Order-Management werden dann die Informationen zu entsprechenden Produkte, die zu dieser Bestellung gehören aus dem Produkt-Management angefordert. Mithilfe der Produktinformationen kann dann im Bezahlbereich ein Endpreis für den Nutzer kalkuliert werden. Die dort generierten Informationen können dann wieder der Website zur Verfügung gestellt werden, damit der Nutzer einen Bezahlvorgang einleiten kann.

\subsection{Monolithen}
Wird nun ein Entwicklerteam damit beauftragt ein System auf Basis dieses Prozesses zu implementieren so kann ein \textbf{monolithischer Ansatz} gewählt werden. Eine ebenfalls vereinfachte Architektur könnte für den oben beschreibenen Prozess folgendermaßen aussehen:

\begin{figure}[h]
	\centering
	\includegraphics[width=1.0\linewidth]{img/monolitische_architektur.png}
	\caption[monolitische Architektur]{Abbildung des Prozesses mithilfe eines monolithischen Ansatzes\\Quelle: Eigen}
	\label{fig:monolithic_arch}
\end{figure}

Die in \vref{fig:monolithic_arch} beschriebene Architektur hat einige Eigenschaften, welche sie zu einer monolithischen Architektur werde lassen. Dazu gehört unter anderem die Tatsache, dass der Prozess in verscheidene Komponenten oder Module unterteilt wird, welche alle zusammen die Applikation bilden. Zusätzlich greifen alle Komponenten, da sie als eine Einheit deployt werden auf dieselbe Datenbank zu. Sowohl die Tatsache, dass es nicht in einzelne Services sonder Komponenten unterteilt wurde und das sich eine gemeinsame Datenbank geteilt wird stellen Merkmale dar, welche auf eine monolithische Architektur schließen lassen. Ein Vorteil dieses Ansatzes ist, dass die Kommunikation zwischen den verscheidenen Komponenten sehr einfach ist, da diese meistens auch logisch als \textit{ein Programm} ablaufen. So kann das Order-Management Daten anfordern indem es eine Methode im Produkt-Management aufruft. Ein \textbf{Nachteil} dieses Ansatzes besteht aber darin, dass eine sehr starke Kohäsion und Abhängigkeit von der spezifischen Implementierung einer Komponente besteht. So kann beispielsweise das Order-Management nur ausgestausch werden, wenn unter viel Aufwand auch alle anderen Komponenten in dem Monolithen angepasst werden. 

\begin{definition}[Monolith]
	Monolithen lassen sich anhand folgender Eigenschaften definieren: \autocite[S. 3]{microservice_enterprise}
	\begin{enumerate}
		\item Monolithen werden als einzelne Einheit entworfen, entickelt und deployt. Dies bringt mit sich, dass sie oftmals eine enorme Komplexität erreichen, welche nicht gut zu überblicken ist.
		\item Die einzelnen implementierten Businessfunktionen können nicht einzeln skaliert oder aktualisiert werden. Alle Komponenten sind also an zentrale Deployments des gesamten Monolithen gebunden, auch wenn nur eine Komponente aktualisiert werden muss.
		\item Die initiale Wahl einer Programmiersprache kann später nichtmehr oder nur sehr schwer wieder geändert werden, da jede einzelne Folgeentscheidung auch auf Basis dieser Wahl getroffen wurde.
		\item Bei Instabilität einer einzelnen Komponenten besteht Gefahr, dass die gesamte Applikation einen Fehler erleidet und nichtmehr funktionsfähig ist.
	\end{enumerate}
\end{definition}

\subsection{SOA und ESB}

Eine Lösung für die Nachteile, die ein Monolith mit sich bringt sollte mithilfe der \ac{SOA} erreicht werden. Die \ac{SOA} versucht die große, schwer skalierbare und stark zusammenhängende Einheit eines Monolithen aufzubrechen in kleine \enquote{self-contained} Services. Diese Services haben einen klar definierten Aufgabenbereich und besitzen ein wohldefiniertes Interface, welches \textbf{unabhängig} von der unterliegenden Implementierung ist. Dies löst bereits mehrere Probleme, welche in einem Monolithen vorhanden waren. So kann nun durch die lose Kopplung zwischen den Services (Komponenten in dem Monolithen) eine individuellere Skalierbarkeit erreicht werden. Jetzt stehen aber nicht mehr nur ein zentraler Endpunkt wie bei dem Monolithen zur Verfügung der von einem Client angesprochen werden kann. Es kommt auch die Frage auf, wie bestimmt werden kann, welcher Instanz eines Service angesprochen wird, wenn dieser in mehreren Replikaten vorliegt. Um unter anderem diese zwei wichtigen Punkte zu lösen wird die \ac{SOA} meistens nur in Verbindung mit einem \ac{ESB} verwendet.

Unter einem \ac{ESB} kann man sich vereinfacht eine Art intelligenten Load-Balancer vorstellen. Zu den klassischen Aufgaben eines \ac{ESB} gehören unter anderem die Weiterleitung der Anfragen eines Clients zu den richtigen Services. Der Grund warum es sich bei dem \ac{ESB} um einen Art intelligenten Load-Balancer handelt ist, weil er zusätzlich die Fähigkeit besitzt einzelne Services zu zusammengesetzten logischen Einheiten zu kombinieren. Wie diese Services kombiniert werden obliegt dabei dem Implementierenden, welcher es auf Basis des zu implementirenden Prozesses entscheiden kann. Zusätzlich können in einem \ac{ESB} noch Funktionen wie beispielsweise Authenthifizierung von Clients oder auch Monitoringfunktionen eingebaut werden.\\
Das oben eingeführte Beispiel könnte umgesetzt in einer \ac{SOA} folgendermaßen umgesetzt werden:
\begin{figure}[]
	\centering
	\caption{TODO: Hier kommt noch das Bild der SOA Arch hin}
\end{figure}

Diese nächste \enquote{Entwicklungsstufe} auf dem Weg zur Microservicearchitektur lässt sich also mithilfe folgender Eigenschaften definieren:

\begin{definition}[SOA und ESB]
	Im Rahmen der \ac{SOA} ist ein Service mit folgenden Eigenschaften definiert: \autocite[S. 4]{microservice_enterprise}
	\begin{enumerate}
		\item Ein Service ist eine eigenständige Implementierung einer wohldefinierten Businessfunktion, welcher über das Netzwerk erreichbar ist. Sie sind lose gekoppelt und verfügen über ein wohldefiniertes Interface über welches sie nach außen hin ansprechbar sind, somit sind sie implementierungsunabhängig. Services stellen die Grundbausteine innerhalb der \ac{SOA} dar.
		\item Zusammengesetzte Services können mithilfe auf Basis bestehender Services generiert werden und erben alle Eigenschaften, die ein Service auch hat.
		\item Services können dynamisch registriert werden. Es ist oftmals für den Client nicht von relevanz den genauen Ort eines Services zu kennen, da diese im Rahmen einer Service-Registry in Form von Metadaten vorliegen.
	\end{enumerate}
	Dem \ac{ESB} fällt dabei die Rollen eines intelligenten Mittelsmann (\enquote{smart Pipeline}) zu. Er besitzt die Möglichkeit Services zusammenzufassen und sich um die Sichtbarkeit, sowie die zusätzliche Bereitstellung von Funktionen zu kümmern. Er stellt einen zentrale Schnittstelle zwischen den einzelnen Services und der \enquote{Außenwelt} innerhalb der \ac{SOA} dar.
\end{definition}

\subsection{Der letzte Schritt - die Microservicearchitektur}
% TODO: Microservices noch endgültig definieren

%%%%%%%%%%%%%%%%%%%%%%%
%	OBSERVABILITY	  %
%%%%%%%%%%%%%%%%%%%%%%%
\section{Observability}

Der Begriff und der Nutzen von Observability für Services lässt sich anhand der folgenden Zitate sehr gut nachvollziehen.

\begin{quote}
	\enquote{Collecting data is cheap, but not having it when you need it can be expensive.}\autocite[S. 373]{microservice_enterprise} - \textit{\citeauthor{microservice_enterprise}}

	\enquote{Observability is the measure of how well internal states [...] of a system can be inferred by knowledge of its external outputs [...].}\autocite[S. 35]{Yordanova2016} - \textit{\citeauthor{Yordanova2016}}
\end{quote}

Innerhalb einer jeden Technologie wird uns durch Standards \marginpar{CNCF} eröglicht wichtige Einsichten in Services zu erhalten. Wird jedoch darauf verzichtet Daten aus einem Service zu sammeln, so kann es im Fehlerfall die Konsquenz haben, dass Fehler erst garnicht entdeckt oder bemerkt, geschweige denn gelöst werden. Die Observability von Services, welche innerhalb einer Applikation genutzt werden ist also von essenzieller Bedeutung. Nun kommt die Frage auf, was genau ist Observability? Welche Aspekte meines Service muss ich überwachen, um zuversichtlich sien zu können, einen Fehlerfall schnell bemerken zu können und diesen dann durch die gesammelten Daten schnell analysieren zu können?\\
In \citetitle{Sridharan2018} wird beschrieben, wie sich Services in verteilten Systemen observieren lassen. Dazu nimmt der Autor eine Unterteilung in drei Säulen vor.

\begin{definition}[Die drei Säulen der Observability]\autocites[Chapter 4: Three Pillars of Observability]{Sridharan2018}[S. 373f]{microservice_enterprise}
	Um erfolgreich auf interne Zustände schließen zu können, müssen Daten existieren auf deren Basis Schlussfolgerungen gezogen werden können. Diese Daten werden aus drei verschiedenen Bereichen erhoben:

	\newcounter{pillarsCounter}

	\begin{enumerate}
		\item \textbf{Logging:}\\
		Das Zeil des Loggings besteht darin, Ereignisse zu dokumentieren. Es ist dabei nicht von Bedeutung, um welche Events es sich handelt. Zusätzlich zu den eigentlichen Daten des Ereigisses, welche unterschiedlichster Art sein können, ist es möglich Metadaten zu loggen, welche dem Ereigis zusätzlichen Kontext verleihen oder einen sonstigen Mehrwert bieten. Dazu gehören z.B. Information zum Zeitpunkt des Ereigisses (\textit{timestamp}), Ergebnis des Ereigisses (\textit{status}), usw. 
		
		Ein Vorteil des Loggings bestehen unter anderem darin, dass diese extrem einfach zu generieren sind. Ein Nachteil des Loggings besteht darin, das Logs an sich keine zusätzliche Aussagekraft haben, außer ihrem direkten Inhalt. Es ist schwer rein anhand von Logs ein komplexes Fehlerbild in einem verteilten System zu finden und korrekte Maßnahmen zur Behebung zu treffen.
		\item \textbf{Metriken:}\\
		Metriken stellen eine numerische representation von Daten dar, welche über bestimmte Zeitintervalle gemessen wurden. Metriken können also als Indikator verwendet werden, wie gut oder schlecht ein Services performt. Populäre Metriken wären z.B. die Anzahl der Anfragen, die ein Service in einer Minute verarbeitet, oder die durschnittliche Antwortzeit eines Service (seine Latenz). Das zweite Beispiel zeigt sogar ein Zusammenspiel zwischen zwei Säulen auf. Diese Metrik kann als abgeleitete Eigenschaft aus Daten des Loggings interpretiert werden, da die Latenz als Differenz zwischen dem Zeitpunkt der Anfrage und dem Zeitpunkt der Antwort ist.
		\setcounter{pillarsCounter}{\value{enumi}}
	\end{enumerate}

	\textbf{Logging} kann also als datenaggregierende Säule angesehen werden. \textbf{Metriken} hingegen nutzen (unter anderem) die aggregierten Daten, um neue wertvollere Informationen abzuleiten. Die Kombination zwischen diesen beiden Säulen bietet einem Aussenstehenden einen guten Einblick in den internen Zusatand \textit{eines} Services. Wie bereits bei der Defintion eines Microservices erwähnt ist es aber nicht unüblich hunderte kleine Services beobachten zu müssen. Es wäre allerdings sehr unpraktisch nur die einzelnen Services in Isolation zu betrachten, da es sich um stark vernetzte und voneinander abhängige Services handelt. Deshalb existieren noch eine dritte Säule, um auch serviceübergreifend Daten zu aggregieren und diese zu verwertbaren Informationen umzuwandeln.

	\begin{enumerate}
		\setcounter{enumi}{\value{pillarsCounter}}
		\item \textbf{Tracing:}\\
		Betrachtet man die bisherigen Säulen, so lässt sich ein Trend auf der Betrachtungsebene feststellen. Das Logging fokussiert sich auf die Sammlung der Daten einzelner Ereignisse, der kleinsten Einheit. Die Metriken arbeiten nun ereignisübergreifend und wandeln die Daten zu nutzbaren Informationen um, damit Einsichten in das Verhalten eines Services vorhanden ist - die mittlere Einheit. Der nächste logische Schritt wäre nun, da Services in einem verteilten System miteinander vernetzt sind, eine Anfrage von Beginn bis zur endgültigen Antwort serviceübergreifend zu observieren, quasi eine End-to-End Betrachung. Genau diese Betrachtung wird mithilfe von Tracing ermöglicht. Eine \textit{Trace} (\enquote{Spur}) ist eine Liste zusammenhängender, verteilter Eregnisse, welche als Repräsentant einer End-to-End Anfrage angesehen werden können.
	\end{enumerate}

\end{definition}
Mithilfe dieser drei Säulen ist es nun einem Aussenstehenden möglich auf \enquote{die inneren Zustände eines Systems auf Basis seiner Ausgaben} zu schließen. Dazu ein kurzes Beispiel:\\
Es wird ein Fehler bei einer Anfrage regisitriert. Der dafür zuständige Mitarbeiter kann nun mithilfe des \textbf{Tracings} feststellen um welches Request es sich handelt. Dabei erkennt er, welche Services als Ursache für den Fehler infragekommen. Im nächsten Schritt kann er sich einzelne \textbf{Metriken} der Services ansehen, um eventuell eine Anomalie festzustellen. Im letzten Schritt kann er durch ansehen der \textbf{Logs} eine genaue Fehlerursache feststellen.


\section{Governance}

Nun können Services observiert werden und ein Unternehmen kann sicherstellen, dass Fehler gefunden werden können, wenn sie auftreten. Nun steht es vor der Aufgabe einen identifizierten Fehler an das richtige Team weiterzuleiten, sodass dieser schnellstmöglich behoben werden kann. Der Vorgang des Findens des Veranwtortlichen ist von immenser Bedeutung, da dort auch die Expertise für einen bestimmten Service liegt. Dies stellt einen wichtigen Teilbereich der Governance dar. Eine klare Defintion der Zustädnigkeiten gestaltet sich aber bei größeren Unternehmen und einer hohen Anzahl an Services immer schwieriger. Im folgenden Abschnitt wird versucht Grundelgende Bestandteile von Microservice Governance herauszuarbeiten, sodass klar ist, was darunter zu verstehn ist.

In \citetitle{microservice_enterprise} wird beschrieben, dass viele Konzepte, welche für die Governance mit \ac{SOA} entwickelt wurden auch auf Microservices anwendbar sind. Um zu verstehen, worum es sich bei Governance überhaupt handlet wird zuerst eine Wortdefinition genutzt:

\begin{definition}[Governance] Dazu wird zunächst eine Defintion aus dem Camebrigde Wörterbuch entnommen:
	\enquote{the way that organizations or countries are managed at the highest level, and the systems for doing this}\autocite{camebdrigeGovernance}
\end{definition}

Bei Governance handelt es sich also um Möglichkeiten, wie Organisationen verwaltet werden. Nun stellt sich die Frage, wie sich Governance in der IT abbilden lässt und was genau es im Bezug auf Microservice Governance bedeutet. Software und insbesondere Microservices müssen verwaltet werden. Vor allem bei der hohen Anzahl an Microservice ist eine klare Form der Verwaltung wichtig. Um Services erfolgreich verwalten zu können sind einige Elemente zentral zu verwalten. Es kann allerdings nicht alles zentral verwaltet werden, wie es bei Monolithen der Fall war. Eine zentrale Verwaltung bietet sich genau dann an, wenn ein fest vorgeschriebenes Toolset in einer (oder wenigen) Programmiersprachen genutzt wird. Bei Microservices steht es dem implementierenden Team frei, die Technologie zu wählen, die ihnen am passensten erscheint für den Anwendungsfall. Dies führt in der Governance allerdings zu Problemen, da es für ein Unternehmen nicht möglich ist, für jede mögliche Kombination von Tools entsprechende Standards festzulegen. Deshalb gibt es die sogenannte \textit{dezentralisierte Gouvernacen}, damit Unternehmen einen Teil zentral verwalten können, und andere wichtige Teile dezentral von den Entwicklerteams entschieden werden können.

\begin{definition}[Dezentralisierte Governance]
	Dezentrale Governance bedeutet, dass Services in folgenden Punkten nicht zentral gesteuert werden, sondern von den Entwickerteams übernommen werden:\autocites[S. 154]{microservice_enterprise} [Decentralized Governance]{FowlerMicrservices}
	
	\begin{itemize}
		\item \textbf{Entwurf eines Services:}\\
		Der Entwurf eines Services wird nicht zentral gesteuert. Dies steht ganz im Rahmen der Defintion eines Microservice, da die Entscheidung über die sinnvollsten Technologien für die Aufgabe bei den Entwicklern liegen soll. Es gilt bei dem Design bereits darauf zu achten, welche Schnittstellen und Abhängigkeiten der eigene Service hat, um später schnelle Fortschritte während der Entwicklung machen zu können.
		\item \textbf{Entwicklung eines Services:}\\
		Auch die Entwicklung eines Services muss vom zuständigen Team überwacht werden. Dies ermöglicht schnellere Entscheidungen und einfachere Prozesse. So muss auch bei Problemen nicht ein Umweg über eine höhere Instanz gewählt werden, sondern es kann direkt mit dem zuständigen Team der Abhängigkeit gesprochen werden.
		\item \textbf{Deployment eines Services}:\\
		Befindet sich ein Service nun in der Entwicklung ist es wichtig schnell einen möglichst automatisierten Ansatz für das Deployment des Services aufzubauen. Auch hier ist es von Vorteil, dass es eine dezentral organisierte Komponente ist, da aufgrund der freien Wahl der Programmiersprache und der Technologie entsprechendes Tooling verwendet werden muss. Auch kommt es immer auf die Art und den Inhalt es Services an, welcher auch Anforderungen an das Deployment stellen kann.
	\end{itemize}
	Im Gegensatz zu diesen Elementen gibt es auch in der dezentralen Governance Elemente, welche zentral verwaltet werden müssen. Dazu zählen:

	\begin{enumerate}
		\item die Observability
		\item eine Service-Registry und Service-Discovery
		\item das API-Management und
		\item die Entwicklungszyklen
	\end{enumerate}

	
\end{definition}

Dieses Konzepte der dezentralen Verwaltung von servicespezifischen Aufgaben ermöglicht es den Teams die Idee von Microservices umzusetzen. Sie haben die Chance ohne Einschränkungen seitens des Unternehmens den Technologieverbund zu wählen, welcher ihnen für den Anwendungsfall am sinnvollsten erscheint. Trotz dieser Freiheit entsteht kaum ein Tradeoff dabei, da weiterhin zentrale Komponenten zur Kontrolle und Steuerung exisiteren, welche mit jedem Service kompatibel sind. Dies wird unter anderem durch das verwenden von offenen Standards ermöglicht, sodass eine große Reihe an Tools miteinander kompatibel sind.

Es lässt sich also feststellen, dass trotz der dezentralen Natur von Microservice einige Komponenten sinnvollerweise zentral angelegt werden.

\section{Betrachtung bestimmter Gesichtspunkte im Rahmen der Governance und Observability}

Viele Konzepte, welche für Microservices essenziell sind stellen mit einem strengen Blick auf die Governance und die Observability eine Herausforderung dar. So werden im folgenden Teil einige dieser Konzepte vorgestellt und die damit verbundenen Probleme erläutert. Diese Probleme müssen mithilfe von Tooling oder eventuell sogar Designentscheidungen während der Softwareentwicklung verhindert werden.

\subsection{Design for Failure}
Bei Design for Failure \autocite{FowlerMicrservices} handelt es sich um ein Kernprinzip, welches bereits von \citeauthor{FowlerMicrservices} in seiner Zusammenstellung zu dem Architekturprinzip Microservice in \enquote{\citetitle{FowlerMicrservices}} vorgestellt wurde. Die Idee hinter diesem Prinzip ist, dass Fehler immer, erst Recht im Softwareumfeld unvermeidbar sind. Deshalb müssen 


Design for Failure, aber wie bekomme ich es mit? Wie reagiere ich im Fehlerfall? Was muss getan werden, und wie groß sind die Auswirkungen wenn einer meiner tausend Microservices ausfällt? Diese und viele weitere Fragen müssen sich DevOps und Software-Engerneering Teams oftmals stellen, wenn sie sich in einer dezentralisierten Microservicearchitektur befinden. Es ist oftmals ein tiefes Verständnis des Zusammenspiels der einzelnen Services nötig um heruaszufinden, ob ein Fehler von eigenen Service kommt, oder ob es aufgrund eines Fehlers in einem anderen Service kommt. Das ultimative Ziel ist es dabei herauszufinden was das Problem ist und dieses auch schnellstmöglich zu beheben, sodass für den Endnutzer keine Merkbaren folgen auftreten. Microservices folgen dem Prinzip \enquote{Design for Failure}, sodass eine Recovery möglich ist und ein operatives Business aufrecht erhalten werden kann. Trotz dieser hervorragenden Prämisse reicht ein \enquote{Design for Failure} alleine nicht aus. \enquote{Ein System ist nur so schlau wie das schwächste seiner Bestandteile}. Microservices mit dem Ziel kleiner dezentralisierter Services, welche einen spezialisiert sind auf eine bestimmte aufgabe innerhalb eines BusinessProzesses haben ironischerweise eine ihrer größten Schwäche in der Kommunikation miteinander. Es müssen Standards etabliert werden, sodass ServiceOwner einen wartbaren und funktionsfähigen Service entwickeln können. Es muss einen Software-Engeneer an einer Stelle ein Fehler unterlaufen und aufgrund der starken Kohäsion und Abhängigkeit der einzelnen Services unterneinander kann eine Reihe wichtiger Businessfunktionen davon betroffen sein.

Kein Problem - \enquote{Design for Failure}. Ein Service fällt aus und ist darauf ausgelegt sich selbst wieder zu reaktivieren. Alle anderen Services können einen ausfall händeln. Soweit die Theorie. In der Praxis ist das leider nur allzuoft nicht der Fall. Innerhalb der Microservice-Governance steht der Aspekt der dezentralisierten Entscheidungsfindung im Vordergrund. Das bedeutet, dass jedes Team das fachliche und unternehmerische Know-How zugesprochen wird das beste Tool und die beste Technologie für die von ihnen zu lösende Aufgabe zu wählen. Fängt ein Team nun an diese Aufgabe zu lösen, so benötigt er oftmals Informationen aus anderen Microservices, um seine Aufgabe zu erfüllen. 


\begin{center}
	\includegraphics[width=0.55\linewidth]{img/service_dependency.png}
\end{center}

Es entsteht also eine Abhängigkeit zwischen zwei Microservices. Ebendieser angefragte Microservice benötigt aber wiederum einen anderen Service, um die angeforderte Information generieren zu können. Es entsteht also schon eine Kette von Abhängigkeiten. Was passiert, wenn ein Glied dieser Kette einen Fehler wirft? 

\Large Hier kommt ein Bild von einem Fehler in einer Microsericekette

\normalsize

Kein Problem - \enquote{Design for Failure}. Ein Architekt muss in der Planung seines Services die Möglichkeit haben, das Risiko und die Auswirkungen einen Ausfalls sowohl seines eigenen Services, als auch seiner Abhängigkeiten einschätzen zu können. Diese Einschätzung sollte auf Daten basieren, welche sowohl Information bisheriger Ausfälle und deren Ursachen enthalten als auch Ausblicke geben können auf den aktuellen Stand und eventuelle zukünftige Ausfälle.

Wie bereits Eingangs erwähnt ist bei der Betrachtung dieses Prinzips die Brille der Governance und Observability aufgesetzt. Es ist ein wichtiger Bestandteil die Services innerhalb einer Unternehmensarchitektur so aufzubauen, dass diese trotz eins Fehlers ordnungsgemäß weiterlaufen und die Fähigkeit zur Recovery besitzen. Dieses Prinzip ist sogar dafür Veranwtortlich, dass Microservicepioniere wie Netflix innerhalb der Observability eine eigene, vierte Säule zu etablieren. Dabei handelt es sich um das sogenannte Chaos-Testing. Um dies kurz zu erläutern: Dabei handelt es sich um einen eigenen \enquote{Service}, welcher auf Basis von Chaos-Experimenten produktive Services ausschaltet und so die Recovery ebendieses Services und auch der davon abhängigen Services überprüfen kann. Dies bringt viele Vorteile mit sich und sorgt auch dafür, dass Services gut entwickelt sind. Diese Idee, wie das bewusste Einführen von Fehlern die generelle Qualität von Services verbessern kann, wird näher in \citetitle{AntifragileOrganization}\autocite{AntifragileOrganization} beschrieben.


\section{Kriterien für gute Gouvernance}

Um im weiteren Verlauf der Arbeit die bereits bestehende und auch die vorgeschlagene Lösung zur Gouvernance und Observability von Microservice miteinander vergleichen zu können müssen einige Kriterien eingeführt werden. Um diese Kriterien zu erarbeiten wird auf vorherige Definition der Gouvernance und Observability geschaut \marginpar{Ref auf G \& O Def.}. Wie bereits zuvor festgestellt wurde besteht die Observability aus drei Säulen - dem \textit{Logging}, dem \textit{Tracing} und den \textit{Metriken}. Ein System welches zum Ziel hat Microservices zu observieren muss also auch in diesem drei Bereichen etwas beitragen. 

Des weiteren spielen Aspekte aus der Microservice Gouvernance eine Rolle für einene Technologie-Stack, welcher wie der hier vorgestellte versucht eine Brücke zwischen diesen beiden Bereichen zu bilden. Ein besonders großer und schwerwiegender Punkt der Microservice Gouvernance stellt eine dezentralisierte Verwaltung dar. Wie bereits \citeauthor{LeanixGouv} in seinem Artikel sagt:
\enquote{
	The main concept of these Microservices are the reusability of assets and tools which can be decentralized. The core theme of a decentralization governance is the concept of building and running it. This decentralized model is best suited for Microservices governances.
}\autocite{LeanixGouv}

Ein Tool welches in diesem Bereich helfen will, muss also als Entscheidungsstütze für dezentralisierte Serviecs funktionieren, um einen mehrwert zu schaffen. \\
Des weiteren muss eine Wertschöpfung aus den gewonnen Daten gezogen werden können. Es stellt also einen wichtigen Faktor dar, alle gesammelten Daten so aufzubereiten, dass es für den Nutzer dieses Tools einen Mehrwert bietet, welcher darüber hinausgeht nur einen Einblick in ebendiese Daten zu erhalten. Beispiele dafür wären Anwendungen im Bereich der \textbf{Predicitve Maintance} oder der \textbf{Anomaly Detection}, usw.

Tools, welche diesen Bereich abdecken müssen also in den folgenden drei großen Kriterien einen Mehrwert gegenüber bestehenden Lösungen bieten:

\begin{enumerate}
	\item Grundelgende Aspekte und Funktionen aus dem Bereich der Microservice Observability müssen abgedeckt sein. Dies beinhaltet die drei Säulen Logging, Tracing und Metriken.
	\item Die dezentralisierte Natur von Microservices muss gefördert werden, sodass ein Tool Entscheidungen - technischer oder unternehmerischer Natur - unterstützen und postiv beeinflussen kann.
	\item Gewonne Daten, egal aus welchen Bereich, müssen einen Mehrwert für den Nutzer darstellen, der über das reine Einsehen der gesammelten Daten hinausgeht.
\end{enumerate}

Auf Basis dieser drei Kriterien sollen nun bestehende Lösungen bewertet werden und mit der eigenen Lösungen verglichen werden, um festzustelle, ob die vorgeschlagene Lösungen
\begin{enumerate}
	\item den definiert Kriterien entspricht und
	\item einen nicht vorhandenen Mehrwert gegenüber bestehenden Lösungen bietet.
\end{enumerate}

Zuletzt ist es sehr wichtig, sollte ein neues Tool gewählt werden darauf zu achten, dass es offenen Standards folgt, um sicherzustelle, dass ein neues (oder bestehendes) System kompatibel ist und mit fremden Tools interagieren kann. Dies ist wichtig, damit auch Unternehmen sicherstellen können, dass der Aufwand, welcher mit der Einführung eines neuen Tolls kommt, ein lohnenswerter ist und eine langfristige Integration aufgrund von Kompatibilität als sinnvoll erachtet werden kann,

\section{Verschiedene Ansätze zum lösen von Gouvernance und Observability}
\subsection{Eigene Ansätze}
\subsection{Ansätze verschiedener Unternehmen}
\begin{itemize}
	\item Elastic \marginpar{Anomaly Detection}
	\item Neo4J \marginpar{ServiceRegistry (Idee für die Arbeit)}
	\item Netflix \marginpar{Master of Microservice}
\end{itemize}
Microservice Gouvernance und Themen der Observability sind bekannte Probleme und wurden schon oft von verschiedenen Organsiationen gelöst. Bewegt man sich im OpenSource bereich so findet sich mit der \ac{CNCF} ein Zusammenschluss, welcher versucht offene Standards für viele elementare Bereiche der Observability und der Gouvernance zu erreichen.

Die \ac{CNCF} definiert sich selbst folgendermaßen:

\enquote{
Cloud native Technologien ermöglichen es Unternehmen, skalierbare Anwendungen in modernen, dynamischen Umgebungen zu implementieren und zu betreiben. Dies können öffentliche, private und Hybrid-Clouds sein. Best-Practises, wie Container, Service-Meshs, Microservices, immutable Infrastruktur und deklarative APIs, unterstützen diesen Ansatz.

Die zugrundeliegenden Techniken ermöglichen die Umsetzung von entkoppelten Systemen, die belastbar, handhabbar und beobachtbar sind. Kombiniert mit einer robusten Automatisierung können Softwareentwickler mit geringem Aufwand flexibel und schnell auf Änderungen reagieren.

Die Cloud Native Computing Foundation fördert die Akzeptanz dieser Paradigmen durch die Ausgestaltung eines Open Source Ökosystems aus herstellerneutralen Projekten. Wir demokratisieren modernste und innovative Softwareentwicklungs-Patterns, um diese Innovationen für alle zugänglich zu machen.
}\autocite{CNCFGithub}

Die \ac{CNCF} vereint also mehrere Tools, welche allen offenen Standards folgen, damit Unternehmen unabhängig von großen Cloudanbietern wie \ac{GCP}, \ac{AWS} oder Microsoft Azure eine Cloud-Infrastruktur aufbauen können. Ein paar Beispiele der \ac{CNCF} zugehörigen Projekte sind treibende, grundlegende Softwarelösungen wie Kubernetes oder OpenShift.

Zusätzlich dient die \ac{CNCF} selbst auch als Organsiation, um verschiedene Industriestandards duchzusetzen. So wurde beispielsweise mithilfe der \textbf{Open Telemetry} ein Standard geschaffen, welcher Toolübergreifen das Sammeln und Auswerten verscheidener Tracingdaten ermöglicht. Dies ermöglicht einem Unternehmen, einen eigenen Stack zusammenzustellen, da es sich darauf verlassen kann, dass die Projekte, welche Teil der \ac{CNCF} sind miteinander kommunizieren und arbeiten können.

\url{https://landscape.cncf.io/category=coordination-service-discovery,service-proxy,service-mesh,observability-and-analysis&format=card-mode}

\section{Auf Basis der kriterien hergeleiteter eigener Ansatz mithilfe eines MicroserviceGraphen}

Innerhalb der Softwareentwicklung stellen sich die oben beschriebenen Probleme oftmals in ähnlicher Form dar. Ein praktisches Beispiel kommt aus dem Bereich der Webentiwcklung, in welchem Bundler ein ähnliches Problem zu bekämpfen haben. Diese besitzen zur Aufgabe Code, welcher in verschiedenen Dateien verteilt ist so zusammenzuführen, das später eine einzige syntaktisch korrete und ausführbare Datei vorliegt. Dabei muss allerdings bedacht werden, dass verschiedene Dateien unterschiedliche Codesegmente aus anderen Dateien importieren können und wiederum eigene Funktionen oder Variablen exportieren können. Dabei zählt es zu den Aufgaben eines Bundlers sicherzustellen, dass ein für die Programmiersprace valider Kontrollflow entsteht. Um dieses Problem zu lösen nutzen Bundler die Möglichkeit, einen Dependency-Graphen auf Basis der Imports und Exports aufzustellen. Dieser wird dann genutzt, um anhand des Startpunktes des Graphen die Zusammenführung der Dateien zu beginnen. Dabei müssen zusätzlich noch weitere Problem gelöst werden, da innerhaln eines Graphen auch circuläre Abhängigkeiten entstehen können, welche aufgelöst werden müssen. Ein Beispiel einer Zirkulären Abhängigkeit kann der \autoref{fig:GraphViz} entnommen werden.

\begin{itemize}
	\item Man es während CI/CD Stuff machen. Auch part von Microservices.
	\item Man kann es danach dynamisch nach Usage discovern lassen.
	\item Man kann einen kombinierten Ansatz wählen, um sowohl initial den Service inklusikve Metadaten sehen und danach die Nutzung ähnlich wie im Epsagon Dashboard zu sehen
\end{itemize}

\begin{figure}[h]
	\centering
	\makebox[\textwidth]{\includegraphics[width=\linewidth]{img/dependency-graph.png}}
	\caption[Dependency Graph]{Visulaisierung eines Dependency-Graphen mit zirkulären Abhängigkeiten. Quelle: \cite[]{GraphViz}}
	\label{fig:GraphViz}
\end{figure}
 All diese Elemente dienen als Grundlage zu der Idee, dass zur Risikoeinschätzung der Auswirkungen eines Fehelerfalls innerhalb einer Microservicearchitektur ein ähnlicher Dependency-Graph eine große Hilfe darstellen würde. Baut man den Graphen so auf, wie in obiger Analogie beschrieben, so erhält man einen Graphen, welcher Anzeigt ob und wie weit bestimmte Microservices miteinenander kommunizieren oder sogar voneinander Abhängigkeit sind. Dies beitet die Möglichkeit wertvolle Einblicke in das \textit{makroskopische} Zusammenspiel der einzelnen Komponenten einer Architektur zu erhalten. Ein Softwarearchitekt hat nun die Möglichkeit bei der Entscheidung über die Einführung eines neuen Service Informationen aus ebendiesem Graphen zu nutzen, um festzustellen ob eine zu Starke Abhängikteit zwischen Services vorhanden ist, oder ob es bei der Entwicklung bereits bestehende Abhängigkeiten gibt.
\marginpar{Hier eventuell bissjen über Kopplung auf Codebassis labern wie das Auswirkungen auf Arhcitektur haben kann} 
Der beschriebene Graph kann also als eine Art Service-Registry angesehen werden, welche Informationen zu der Beziehung zwischen verschiedenen Services beinhaltet. Gleichzeitig wird dadurch vor allem der Bereich der Microservice-Gouvernance betreten, wo eine Service-Registry eine zentrale Komponente darstellt. 

Zusammenfassend lässt sich das Konzept also folgendermaßen beschreiben: \\
Eine Schnittstelle zwischen Microservice-Gouvernance und -Observability wird mithilfe eines Dependency-Graphen gebildet. So spielen Aspekte einer Service-Registry und Elemente des Distributed-Tracing eine große Rolle in diesem Ansatz. 
\chapter{Idee und POC eines Microservice-Dependency Graphen}

Nun, da Grundlegende Begriffe geklärt wurden, werden Kriterien heruasgeabreitet, welche zur spätere Einschätzung und Bewertung eines vorgeschlagenen Ansatzes diesen sollen. Gleichzeitig werden diese Kriterien genutzt, um beretis bestehende Ansätze zu untersuchen und auf ihre Anwendbarkeit zu überprüfen.

Im Rahmen der Observability stehen drei große Säulen im Fokus:
\begin{enumerate}
	\item Logging
	\item Tracing
	\item Metriken
\end{enumerate}

Diese drei Elemente spielen eine wichtige Rolle um eine Lösung im Bereich der Observability liefern zu können. Im Rahmen dieser Arbeit wird vor allem ein Augenmerk auf den Status einer Microservice-Archtitektur gelegt, welche nicht zum Ziel hat, die softwaretechnische Ursache des Problems zu finden, sondern den Service oder den Container zu identifizieren, welcber für den Ausfall veranwtortlich ist. Gleichzeitig gibt es die Anforderung, dass die Abhängigkeiten verschiedener Microservices voneinander dargestellt und in Form von nutzbaren, auswertbaren Daten vorliegen. Hierbei kommt allerdings ein Punkt der Observability zu tragen, da dies, wie später näher erläutert wird auch mithilfe von Tracing erreicht werden kann.\\
Außerdem muss es möglich sein, die gewonnen Daten nutzbar zu machen, um nicht nur feststellen zu können, wecher Service die Ursache des Problems ist, sonder auch, um die Auswirkungen auf andere Services feststellen zu können. Es muss also ein einfacher Weg geschaffen werden, damit die Auswirkungen eines Fehlers schnell festgestellt werden können. 

\chapter{Bewertung und Einschätzung der Arbeit, sowie Ausbilck auf künfitge Arbeiten}
%	Literaturverzeichnis
\clearpage
\ihead{}
\printbibliography[title=Literaturverzeichnis]
\cleardoublepage

% Der Anhang beginnt hier - jedes Kapitel wird alphabetisch aufgezählt. (Anhang A, B usw.)
% \appendix
% \ihead{\appendixname~\thechapter} % Neue Header-Definition

% Ehrenwörtliche Erklärung ewerkl.tex einziehen
\input{ewerkl.tex}


\end{document}
